\chapter{Flow and Transport in Lateral Cavities with Aquatic Vegetation: a numerical approach}
\label{chap:art4}
In this chapter, the main topic of this dissertation is developed. The effects of vegetation on the hydrodynamics and the mass exchange between the main channel/dead zone are investigated. The objective of this paper was to describe and possibly find a threshold on the behaviour of the dead zone given a certain density level.

\addcontentsline{toc}{section}{Abstract}
\section*{Abstract}
Lateral cavities are regions attached to rivers affect the flow by creating a dead water zone. These regions reduce the flow velocity increasing the deposition of sediment and the temporary storage of polluted materials, which favours the growth of aquatic vegetation. The effect of this vegetation growth was studied using different vegetation densities in a Large Eddy Simulation (LES). The vegetation drag was represented by a porous media calculated with the Darcy-Forchheimer model. This numerical model showed that the hydrodynamics of the flow can present different patterns and phases for a vegetation density $0<a(\%)<10.656$. Furthermore, the occurrence of a secondary circulation was found where it normally would not occur for a non-vegetated scenario.

\noindent\textbf{Keywords:}: Lateral Cavity; Aquatic Vegetation; Mass Exchange; Computational Fluid Dynamics (CFD).

\section{Introduction}

\addcontentsline{toc}{section}{References}
\printbibliography[segment=\therefsegment,heading=subbibliography, title={References}]