\chapter{Conclusion and Recommendations}
\label{chap:conclusion}

The objective of this study was the description of the hydrodynamics and mass exchange in dead waters. In the present study, numerical experiments were performed to describe the flow in groynes and lateral cavities.

First, a literature review of the flow conditions and methods used to describe the flow was performed which showed gaps in knowledge to be fulfilled. Second, the first description of the groyne flow was presented along the study of tracer fields, which showed that the mass exchange between the DZ and the main channel is not unique and in fact occurs at two different rates depending on the elapsed time. Third, the first description of lateral cavities was presented, in this paper we described the flow using a model developed in a commercial software. Forth, the description of lateral cavities was further developed using another approach to the turbulence fields and an open source package. Finally, the effects of vegetation in lateral cavities was described.

The differences in modelling groynes and lateral cavities are significant. The study of groynes imply a periodicity that must be fulfilled by two means: a) a series of groyne fields or b) a pair of cyclic/periodic surfaces. The difficulties inhered by both methods rely on the high computational cost, as in the option a) the domain is extensive and the meshing process is harder and usually means in a lost of detail as a refined mesh becomes prohibitive. On another hand, second option provides a more accurate description of the flow as the meshing can be concentrated only on one groyne field, although the implementation of a periodic surface in a zone of high mixture implies in a requirement for a small cell size specially in the groyne head, region of intense vortex shredding. In contrast to groynes, cavities do not require repetition and can be represented in a simple inlet/outlet scheme, this implies in reduced computational costs.

For both studies, as turbulence is the main phenomena in DZ the selection of the turbulence model is primordial to the model. Through the investigation process it was noticed that the Reynolds Averaging Navier-Stokes gives an approximation of the flow, we analysed multiple models that rely on this spectra and only the k-$\omega$ SST model was suitable for this kind of flow. Another hybrid models such as the Detached Eddy Simulation was a further improvement to the description of turbulence. Although, some structures were clearer when the Large Eddy Simulation was introduced, as the instantaneous flow was the considered.
