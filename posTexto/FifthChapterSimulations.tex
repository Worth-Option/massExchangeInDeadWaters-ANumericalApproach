\noindent In this appendix, the code in the chapter: The Effects of Vegetation Density Upon Flow and Mass Transport in Lateral Cavities. A copy of this configuration will also be available at the \href{https://github.com/Worth-Option/massExchangeInDeadWaters-ANumericalApproach}{Github} repository linked in the conclusion.

\section{File Structure}
\noindent
The file structure of the script is shown bellow:
\\
\dirtree{%
.1 /.
.2 0.orig\DTcomment{Initial Boundary Conditions Folder}.
.3 nut\DTcomment{Boundary conditions of turbulent viscosity}.
.3 p\DTcomment{Boundary conditions of pressure}.
.3 tracer\DTcomment{Boundary conditions of the inert scalar}.
.3 O\DTcomment{Boundary conditions of velocity}.
.2 constant\DTcomment{Mesh and General information about the simulation}.
.3 fvOptions\DTcomment{Configuration of the porous media}.
.3 g\DTcomment{Gravity}.
.3 transportProperties\DTcomment{Fluid characteristics}.
.3 turbulenceProperties\DTcomment{Turbulence settings}.
.4 boundaryData\DTcomment{Pre-calculated fields for the inlet}.
.5 inlet.
.6 points.
.6 0.
.7 k.
.7 L.
.7 nut.
.7 nuTilda.
.7 omega.
.7 p.
.7 R.
.7 U.
.2 system\DTcomment{Main configuration folder}.
.3 blockMeshDict\DTcomment{Mesh configuration}.
.3 controlDict\DTcomment{Simulation Control}.
.3 decomposeParDict\DTcomment{Configuration of the parallelisation of the grid}.
.3 fvSchemes\DTcomment{Configuration of the used numerical schemes}.
.3 fvSolution\DTcomment{Configuration of the solver}.
.3 setFieldsDict\DTcomment{Set of the initial tracer fields}.
.3 topoSetDict\DTcomment{Mesh manipulation}.
.3 totalTKE.
.2 allClear.
.2 mesh\DTcomment{Bash script to aid the creation of the mesh}.
.2 ramCache\DTcomment{Bash script to clear the memory cache}.
.2 reconstructParParallel\DTcomment{Union of the parallel cases into a single directory}.
.2 x\#\#\#.foam\DTcomment{Header file for the visualisation software (Paraview)}.
}

\section{0.orig/nut}
\begin{lstlisting}
/*--------------------------------*- C++ -*----------------------------------*\
| =========                 |                                                 |
| \\      /  F ield         | OpenFOAM: The Open Source CFD Toolbox           |
|  \\    /   O peration     | Version:  v1912                                 |
|   \\  /    A nd           | Website:  www.openfoam.com                      |
|    \\/     M anipulation  |                                                 |
\*---------------------------------------------------------------------------*/
FoamFile
{
    version     2.0;
    format      ascii;
    class       volScalarField;
    location    "0";
    object      nut;
}
// * * * * * * * * * * * * * * * * * * * * * * * * * * * * * * * * * * * * * //

dimensions      [0 2 -1 0 0 0 0];

internalField   uniform 0;

boundaryField
{
    inlet
	{
		type       		timeVaryingMappedFixedValue;
		setAverage      false;
		perturb         0;
    }
    outlet
    {
        type            calculated;
        value           uniform 0;
    }
    bottom
    {
        type            nutUSpaldingWallFunction;
        value           uniform 0;
        maxIter			100;
        tolerance		1e-07;
    }
    lateralWall
    {
        type            nutUSpaldingWallFunction;
        value           uniform 0;
        maxIter			100;
        tolerance		1e-07;
    }
    freeSurface
    {
        type            zeroGradient;
    }
    farField
    {
		type			zeroGradient;
	}
}


// ************************************************************************* //

\end{lstlisting}

\section{0.orig/p}
\begin{lstlisting}
/*--------------------------------*- C++ -*----------------------------------*\
| =========                 |                                                 |
| \\      /  F ield         | OpenFOAM: The Open Source CFD Toolbox           |
|  \\    /   O peration     | Version:  v1912                                 |
|   \\  /    A nd           | Website:  www.openfoam.com                      |
|    \\/     M anipulation  |                                                 |
\*---------------------------------------------------------------------------*/
FoamFile
{
    version     2.0;
    format      ascii;
    class       volScalarField;
    location    "0";
    object      p;
}
// * * * * * * * * * * * * * * * * * * * * * * * * * * * * * * * * * * * * * //

dimensions      [0 2 -2 0 0 0 0];

internalField   uniform 0;

boundaryField
{
    inlet
    {
        type            zeroGradient;
    }
    outlet
    {
        type            fixedValue;
        value           uniform 0;
    }
    bottom
    {
        type            zeroGradient;
    }
    lateralWall
    {
        type            zeroGradient;
    }
    freeSurface
    {
        type            zeroGradient;
    }
    farField
    {
		type			zeroGradient;
	}
}


// ************************************************************************* //

\end{lstlisting}

\section{0.orig/tracer}
\begin{lstlisting}
/*--------------------------------*- C++ -*----------------------------------*\
| =========                 |                                                 |
| \\      /  F ield         | OpenFOAM: The Open Source CFD Toolbox           |
|  \\    /   O peration     | Version:  v1912                                 |
|   \\  /    A nd           | Website:  www.openfoam.com                      |
|    \\/     M anipulation  |                                                 |
\*---------------------------------------------------------------------------*/
FoamFile
{
    version     2.0;
    format      ascii;
    class       volScalarField;
    location    "0";
    object      tracer;
}
// * * * * * * * * * * * * * * * * * * * * * * * * * * * * * * * * * * * * * //

dimensions      [0 0 0 0 0 0 0];

internalField   uniform 0;

boundaryField
{
    inlet           
    {
        type			zeroGradient;
    }
    outlet          
    {
        type			zeroGradient;
    }
    bottom
    {
        type            zeroGradient;
    }
    lateralWall
    {
        type            zeroGradient;
    }
    farField
    {
        type            zeroGradient;
    }
    freeSurface
    {
        type            zeroGradient;
    }
}


// ************************************************************************* //

\end{lstlisting}

\section{0.orig/U}
\begin{lstlisting}
/*--------------------------------*- C++ -*----------------------------------*\
| =========                 |                                                 |
| \\      /  F ield         | OpenFOAM: The Open Source CFD Toolbox           |
|  \\    /   O peration     | Version:  v1912                                 |
|   \\  /    A nd           | Website:  www.openfoam.com                      |
|    \\/     M anipulation  |                                                 |
\*---------------------------------------------------------------------------*/
FoamFile
{
    version     2.0;
    format      ascii;
    class       volVectorField;
    location    "0";
    object      U;
}
// * * * * * * * * * * * * * * * * * * * * * * * * * * * * * * * * * * * * * //

dimensions      [0 1 -1 0 0 0 0];

internalField   uniform (0.101 0 0);

boundaryField
{
    inlet
    {
		type        	turbulentDFSEMInlet;
        delta       	0.021;
        interpolateU	true;
        interpolateL	true;
        interpolateR	true;
        value       	uniform (0.101 0 0);
    }
    outlet
    {
        type            zeroGradient;
    }
    bottom
    {
        type            noSlip;
    }
    lateralWall
    {
        type            noSlip;
    }
    freeSurface
    {
        type            slip;
    }
    farField
    {
		type			slip;
	}
}


// ************************************************************************* //

\end{lstlisting}

\section{constant/fvOptions}
\begin{lstlisting}
/*--------------------------------*- C++ -*----------------------------------*\
| =========                 |                                                 |
| \\      /  F ield         | OpenFOAM: The Open Source CFD Toolbox           |
|  \\    /   O peration     | Version:  v1912                                 |
|   \\  /    A nd           | Website:  www.openfoam.com                      |
|    \\/     M anipulation  |                                                 |
\*---------------------------------------------------------------------------*/
FoamFile
{
    version     2.0;
    format      ascii;
    class       dictionary;
    location    "constant";
    object      fvOptions;
}
// * * * * * * * * * * * * * * * * * * * * * * * * * * * * * * * * * * * * * //

embayment
{
    type            explicitPorositySource;
    active			true;
	selectionMode   cellZone;
	cellZone        embayment;

    explicitPorositySourceCoeffs
    {
        selectionMode   cellZone;
        cellZone        embayment;

        type            DarcyForchheimer;
		
		mu	mu;
        d   (116.62 116.62 4.51E-04);	//Original values d   (116.62 116.62 4.51E-04);
        f   (3.09 3.09 6.08E-03);		//Original values f   (3.09 3.09 6.08E-03);

        coordinateSystem
        {
            origin  (0.25 0.30 0);
            e1      (1 0 0);
            e2      (0 1 0);
        }
    }
}


//************************************************************************* //

\end{lstlisting}

\section{constant/g}
\begin{lstlisting}
/*--------------------------------*- C++ -*----------------------------------*\
| =========                 |                                                 |
| \\      /  F ield         | OpenFOAM: The Open Source CFD Toolbox           |
|  \\    /   O peration     | Version:  v1912                                 |
|   \\  /    A nd           | Website:  www.openfoam.com                      |
|    \\/     M anipulation  |                                                 |
\*---------------------------------------------------------------------------*/
FoamFile
{
    version     2.0;
    format      ascii;
    class       uniformDimensionedVectorField;
    location    "constant";
    object      g;
}
// * * * * * * * * * * * * * * * * * * * * * * * * * * * * * * * * * * * * * //

dimensions      [0 1 -2 0 0 0 0];
value           (0 0 -9.81);


// ************************************************************************* //

\end{lstlisting}

\section{constant/transportProperties}
\begin{lstlisting}
/*--------------------------------*- C++ -*----------------------------------*\
| =========                 |                                                 |
| \\      /  F ield         | OpenFOAM: The Open Source CFD Toolbox           |
|  \\    /   O peration     | Version:  v1912                                 |
|   \\  /    A nd           | Website:  www.openfoam.com                      |
|    \\/     M anipulation  |                                                 |
\*---------------------------------------------------------------------------*/
FoamFile
{
    version 2.0;
    format ascii;
    class dictionary;
    location constant;
    object transportProperties;
}
// * * * * * * * * * * * * * * * * * * * * * * * * * * * * * * * * * * * * * //

	transportModel  Newtonian;
	nu              [ 0 2 -1 0 0 0 0 ]	1E-6;
	mu				[ 1 -1 -1 0 0 0 0 ]	1E-03
	rho             [ 1 -3 0 0 0 0 0 ]	1000;	

// ************************************************************************* //

\end{lstlisting}

\section{constant/turbulenceProperties}
\begin{lstlisting}
/*--------------------------------*- C++ -*----------------------------------*\
| =========                 |                                                 |
| \\      /  F ield         | OpenFOAM: The Open Source CFD Toolbox           |
|  \\    /   O peration     | Version:  v1912                                 |
|   \\  /    A nd           | Website:  www.openfoam.com                      |
|    \\/     M anipulation  |                                                 |
\*---------------------------------------------------------------------------*/
FoamFile
{
    version     2.0;
    format      ascii;
    class       dictionary;
    location    "constant";
    object      turbulenceProperties;
}
// * * * * * * * * * * * * * * * * * * * * * * * * * * * * * * * * * * * * * //

simulationType LES;

LES
{
	turbulence      on;
	LESModel		WALE;
	printCoeffs		on;
	
	delta           cubeRootVol; //since the WALE model does not require damping close to the wall
}

// ************************************************************************* //

\end{lstlisting}

\section{system/blockMeshDict}
\begin{lstlisting}
/*--------------------------------*- C++ -*----------------------------------*\
| =========                 |                                                 |
| \\      /  F ield         | OpenFOAM: The Open Source CFD Toolbox           |
|  \\    /   O peration     | Version:  v1912                                 |
|   \\  /    A nd           | Website:  www.openfoam.com                      |
|    \\/     M anipulation  |                                                 |
\*---------------------------------------------------------------------------*/
FoamFile
{
    version 2.0;
    format ascii;
    class dictionary;
    location system;
    object blockMeshDict;
}
// * * * * * * * * * * * * * * * * * * * * * * * * * * * * * * * * * * * * * //

    // Geometry Parameters
    inletX			0.25;
    channelY		0.30;
    embX			#calc "$inletX + 0.25";
    embY			#calc "$channelY + 0.15";
    outletX			#calc "2*$embX + $inletX";
    depth			0.1;
    
    // Mesh Parameters
    z				40;
    embx			80;
    emby			80;
    
    ioX				40;
    outX			120;
    ioY				120;

	gradingX		1;
	gradingXinv		1;
	gradingY		2;
	gradingYinv		0.5;
	
	embGradingY		2;
	embGradingYinv	0.5;
	
	gradingZ		41;

    scale 1;
    vertices  
    (
      // Bottom Vertices
      (0.00 0.00 0.000)						//0
      ($inletX 0.00 0.000)					//1
      ($embX 0.00 0.000)					//2
      ($outletX 0.00 0.000)					//3
      ($outletX $channelY 0.000)			//4
      ($embX $channelY 0.000)				//5
      ($inletX $channelY 0.000)				//6
      (0.00 $channelY 0.000)				//7
      ($embX $embY 0.000)					//8
      ($inletX $embY 0.000)					//9
		
      // Upper Vertices	
      (0.00 0.00 $depth)					//10
      ($inletX 0.00 $depth)					//11
      ($embX 0.00 $depth)					//12
      ($outletX 0.00 $depth)				//13
      ($outletX $channelY $depth)			//14
      ($embX $channelY $depth)				//15
      ($inletX $channelY $depth)			//16
      (0.00 $channelY $depth)				//17
      ($embX $embY $depth)					//18
      ($inletX $embY $depth)				//19
    );

    blocks  
    ( 	
		hex
		( 6 5 8 9 16 15 18 19)
		embayment
		( $embx $emby $z)
		simpleGrading
		(
			(
				(0.1 0.2 $embGradingY)
				(0.8 0.6 1)
				(0.1 0.2 $embGradingYinv)
			)
			(
				(0.1 0.2 $embGradingY)
				(0.8 0.6 1)
				(0.1 0.2 $embGradingYinv)
			)
			$gradingZ
		)
		
		hex
		( 0 1 6 7 10 11 16 17)
		inlet_channel
		( $ioX $ioY $z)
		simpleGrading
		(
			1
			//(
			//	(0.25 0.3 $gradingX)
			//	(0.50 0.4 1)
			//	(0.25 0.3 $gradingXinv)
			//)
			(
				(0.1 0.2 $gradingY)
				(0.8 0.6 1)
				(0.1 0.2 $gradingYinv)
			)
			$gradingZ
		)

		
	
		hex
		( 1 2 5 6 11 12 15 16)
		middle_channel
		( $embx $ioY $z)
		simpleGrading
		(
			(
				(0.1 0.2 $embGradingY)
				(0.8 0.6 1)
				(0.1 0.2 $embGradingYinv)
			)
			(
				(0.1 0.2 $gradingY)
				(0.8 0.6 1)
				(0.1 0.2 $gradingYinv)
			)
			$gradingZ
		)
		
		hex
		( 2 3 4 5 12 13 14 15)
		outlet_channel
		( $outX $ioY $z)
		simpleGrading
		(
			1
			//(
			//	(0.25 0.3 $gradingX)
			//	(0.50 0.4 1)
			//	(0.25 0.3 $gradingXinv)
			//)
			(
				(0.1 0.2 $gradingY)
				(0.8 0.6 1)
				(0.1 0.2 $gradingYinv)
			)
			$gradingZ
		)
    );

    edges  
    (
    );

    boundary  
    (
	inlet
	{
		type	patch;
		faces
		(
			( 0 7 17 10)
		);
	}
	outlet
	{
		type 	patch;
		faces
		(
			( 3 4 14 13)
		);
	}
	bottom
	{
		type 	wall;
		faces
		(
			( 0 1 6 7)
			( 1 2 5 6)
			( 2 3 4 5)
			( 6 5 8 9)
		);
	}
	lateralWall
	{	
		type 	wall;
		faces
		(
			( 7 6 16 17)
			( 6 9 19 16)
			( 9 8 18 19)
			( 5 15 18 8)
			( 5 4 14 15)
		);
	}
	farField
	{
		type	wall;
		faces
		(
			( 0 10 11 1)
			( 1 11 12 2)
			( 2 12 13 3)
		);
	}
	freeSurface
	{
		type 	wall;
		faces
		(
			( 10 11 16 17)
			( 11 12 15 16)
			( 12 13 14 15)
			( 16 15 18 19)
		);
	}
    );
    mergePatchPairs  
    (
    );

// ************************************************************************* //

\end{lstlisting}

\section{system/controlDict}
\begin{lstlisting}
/*--------------------------------*- C++ -*----------------------------------*\
| =========                 |                                                 |
| \\      /  F ield         | OpenFOAM: The Open Source CFD Toolbox           |
|  \\    /   O peration     | Version:  v1912                                 |
|   \\  /    A nd           | Website:  www.openfoam.com                      |
|    \\/     M anipulation  |                                                 |
\*---------------------------------------------------------------------------*/
FoamFile
{
    version 2.0;
    format ascii;
    class dictionary;
    location system;
    object controlDict;
}
// * * * * * * * * * * * * * * * * * * * * * * * * * * * * * * * * * * * * * //

	application				pimpleFoam;
    startFrom				latestTime;
    startTime				0;
    stopAt					endTime;
    endTime					1000;
    deltaT					1.0E-3;
    writeControl			adjustableRunTime;
    writeInterval			10;
    purgeWrite				0;
    writeFormat				ascii;
    writePrecision			6;
    writeCompression		yes;
    timeFormat				general;
    timePrecision			6;
    graphFormat				raw;
    runTimeModifiable		yes;
    adjustTimeStep			true;
    maxCo					0.90;
    maxDeltaT				0.05;

functions
{
	turbulenceFields1
	{
		type					turbulenceFields;
		libs					("libfieldFunctionObjects.so");
		writeControl			writeTime;
		timeStart       		150;
		fields					(R nuTilda L k I);
	}
	
	Q1 //second invariant of the velocity gradient tensor
	{
		type            		Q;
		libs            		("libfieldFunctionObjects.so");
		timeStart       		150;
		writeControl    		writeTime;
	}

	yPlus1
	{
	    type       				yPlus;
	    libs       				("libfieldFunctionObjects.so");
	    timeStart       		150;
	    writeControl			writeTime;
	}
	
	Co1
	{
		type					CourantNo;
		libs					("libfieldFunctionObjects.so");
		timeStart       		150;
		writeControl			writeTime;
	}

	vorticity1
	{
		type        			vorticity;
		libs        			("libfieldFunctionObjects.so");
		timeStart       		150;
		writeControl    		writeTime;
	}
	
	wallShearStress1
	{
		type            		wallShearStress;
		libs            		("libfieldFunctionObjects.so");
		timeStart       		150;
		writeControl    		writeTime;
	
	}
	
	LambVector1	//cross product of a velocity vector [m/s] and vorticity vector [1/s]
	{
		type        			lambVector;
		libs            		("libfieldFunctionObjects.so");
		libs            		("libfieldFunctionObjects.so");
		timeStart       		150;
		writeControl    		writeTime;
	}
	
	//#includeFunc absUy
	
	UyExtract
	{
		type					components;
		libs            		(fieldFunctionObjects);
		field					U;
		timeStart       		150;
		writeControl			none;
	}
	
	absUy
	{
		type					mag;
		libs            		(fieldFunctionObjects);
		field					Uy;
		result					absUy;
		timeStart       		150;
		writeControl			none;
	}
	
	surfaceInterpolate1
	{
		type            		surfaceInterpolate;
		libs            		(fieldFunctionObjects);
		fields      			((absUy absUySurface));
		timeStart       		150;
		writeControl			none;
	}
	
	velocityInterface
	{
		type            		surfaceFieldValue;
		libs            		(fieldFunctionObjects);
		fields          		(absUySurface);
		operation       		areaIntegrate;
		regionType      		faceZone;
		name            		interface;
		timeStart       		150;
		executeControl  		timeStep;
		executeInterval 		1;
		writeControl    		timeStep;
		writeInterval   		1;
		writeFields     		false;
	}
	
	tracer
	{
		type 					scalarTransport;
		libs					("libsolverFunctionObjects.so");
		enabled					true;
		timeStart				150;
		writeControl			writeTime;
		log						yes;

		nCorr					1;

		// Turbulent diffusivity;
		alphaD					0.001;		// Molecular diffusivity
		alphaDt					1.111;		// Turbulent diffusivity (alphaDt = 1 / Sct)
		
		// Bounds the transported scalar within 0 and 1
		bounded01				true;
		
		//name of field
		field					tracer;
	}
	
	tracerVolAverage
	{
		type            		volFieldValue;
		libs            		("libfieldFunctionObjects.so");
	
		log             		true;
		timeStart				150;
		writeControl			timeStep;
		writeInterval			1;
		writeFields     		true;
			
		regionType      		cellZone;
		name            		porousZone;
		operation       		volAverage;
	
		fields
		(
			tracer
		);
	}
	
	surfaceInterpolateTracer
	{
		type            		surfaceInterpolate;
		libs            		(fieldFunctionObjects);
		fields      			((tracer tracerSurface));
		timeStart       		150;
		writeControl			none;
	}
	
	tracerBottom
	{
		type            		surfaceFieldValue;
		libs            		(fieldFunctionObjects);
		fields          		(tracerSurface);
		operation       		average;
		regionType      		faceZone;
		name            		interfaceBottom;
		timeStart       		150;
		executeControl  		timeStep;
		executeInterval 		1;
		writeControl    		timeStep;
		writeInterval   		1;
		writeFields     		false;
	}
	
	tracerMiddle
	{
		type            		surfaceFieldValue;
		libs            		(fieldFunctionObjects);
		fields          		(tracerSurface);
		operation       		average;
		regionType      		faceZone;
		name            		interfaceMiddle;
		timeStart       		150;
		executeControl  		timeStep;
		executeInterval 		1;
		writeControl    		timeStep;
		writeInterval   		1;
		writeFields     		false;
	}
	
	tracerTop
	{
		type            		surfaceFieldValue;
		libs            		(fieldFunctionObjects);
		fields          		(tracerSurface);
		operation       		average;
		regionType      		faceZone;
		name            		interfaceTop;
		timeStart       		150;
		executeControl  		timeStep;
		executeInterval 		1;
		writeControl    		timeStep;
		writeInterval   		1;
		writeFields     		false;
	}
	
 	generalVariablesAveraging
	{
	    type					fieldAverage;
	    libs       				("libfieldFunctionObjects.so");
	    enabled					true;
	    writeControl			writeTime;
	    timeStart				150;
	    restartOnRestart		false;
	    resetOnOutput			false;
	        
	    fields
	    (
	        U
	        {
	            mean			on;
	            prime2Mean		on;
	            base			time;
	        }

			p
	        {
	            mean			on;
	            prime2Mean		on;
	            base			time;
	        }
               
            Co
            {
               	mean			on;
	            prime2Mean		on;
	            base			time;
	        }
               
            yPlus
            {
               	mean			on;
	            prime2Mean		on;
	            base			time;
	        }
	        
	        turbulenceProperties:R
	        {
	            mean			on;
	            prime2Mean		on;
	            base			time;
	        }
	        
	        vorticity
	        {
	            mean			on;
	            prime2Mean		on;
	            base			time;
	        }
	        
	        lambVector
	        {
	            mean			on;
	            prime2Mean		on;
	            base			time;
	        }
		);
	}
	
	#includeFunc totalTKE
	
	totalTKEAveraging
	{
	    type					fieldAverage;
	    libs       				("libfieldFunctionObjects.so");
	    enabled					true;
	    writeControl			writeTime;
	    timeStart				160;
	    restartOnRestart		false;
	    resetOnOutput			false;
	        
	    fields
	    (
	        totalTKE
	        {
	            mean			on;
	            prime2Mean		on;
	            base			time;
	        }
		);
	}
	
	probes
	{
		type				probes;
		libs				("libsampling.so");
		writeControl		timeStep;
		writeInterval		1;
		setFormat			csv;

		fields
		(
			p U
		);

		probeLocations
		(
			(0.25 0.30 0.05)		//0
			(0.30 0.30 0.05)		//1
			(0.35 0.30 0.05)		//2
			(0.40 0.30 0.05)		//3
			(0.45 0.30 0.05)		//4
			(0.50 0.30 0.05)		//5
		);
	}
	
	meanProbes
	{
		type 				probes;
		libs				("libsampling.so");
		writeControl		timeStep;
		writeInterval		1;
		setFormat			csv;
		timeStart			150;
		
		fields
		(
			pMean UMean pPrime2Mean UPrime2Mean
		);

		probeLocations
		(
			(0.25 0.30 0.05)		//0
			(0.30 0.30 0.05)		//1
			(0.35 0.30 0.05)		//2
			(0.40 0.30 0.05)		//3
			(0.45 0.30 0.05)		//4
			(0.50 0.30 0.05)		//5
		);
	}
	
	genericalPlanes
	{
		type				surfaces;
		libs				("libsampling.so");
		writeControl		onEnd;

		interpolationScheme cell;
		surfaceFormat		raw;

		surfaces
		(
			p00
			{
				type            cuttingPlane;
				planeType		pointAndNormal;

				pointAndNormalDict
				{
					point		(0 0.30 0);
					normal		(0 1 0);
					zone		porousZone;
				}
			}
			p01
			{
				type            cuttingPlane;
				planeType		pointAndNormal;

				pointAndNormalDict
				{
					point		(0 0.33 0);
					normal		(0 1 0);
					zone		porousZone;
				}
			}
			p02
			{
				type            cuttingPlane;
				planeType		pointAndNormal;

				pointAndNormalDict
				{
					point		(0 0.36 0);
					normal		(0 1 0);
					zone		porousZone;
				}
			}
			p03
			{
				type            cuttingPlane;
				planeType		pointAndNormal;

				pointAndNormalDict
				{
					point		(0 0.39 0);
					normal		(0 1 0);
					zone		porousZone;
				}
			}
			p04
			{
				type            cuttingPlane;
				planeType		pointAndNormal;

				pointAndNormalDict
				{
					point		(0 0.42 0);
					normal		(0 1 0);
					zone		porousZone;
				}
			}
			p05
			{
				type            cuttingPlane;
				planeType		pointAndNormal;

				pointAndNormalDict
				{
					point		(0.28 0 0);
					normal		(1 0 0);
					zone		porousZone;
				}
			}
			p06
			{
				type            cuttingPlane;
				planeType		pointAndNormal;

				pointAndNormalDict
				{
					point		(0.32 0 0);
					normal		(1 0 0);
					zone		porousZone;
				}
			}
			p07
			{
				type            cuttingPlane;
				planeType		pointAndNormal;

				pointAndNormalDict
				{
					point		(0.35 0 0);
					normal		(1 0 0);
					zone		porousZone;
				}
			}
			p08
			{
				type            cuttingPlane;
				planeType		pointAndNormal;

				pointAndNormalDict
				{
					point		(0.38 0 0);
					normal		(1 0 0);
					zone		porousZone;
				}
			}
			p09
			{
				type            cuttingPlane;
				planeType		pointAndNormal;

				pointAndNormalDict
				{
					point		(0.42 0 0);
					normal		(1 0 0);
					zone		porousZone;
				}
			}
			p10
			{
				type            cuttingPlane;
				planeType		pointAndNormal;

				pointAndNormalDict
				{
					point		(0.45 0 0);
					normal		(1 0 0);
					zone		porousZone;
				}
			}
			p11
			{
				type            cuttingPlane;
				planeType		pointAndNormal;

				pointAndNormalDict
				{
					point		(0.48 0 0);
					normal		(1 0 0);
					zone		porousZone;
				}
			}
			p12
			{
				type            cuttingPlane;
				planeType		pointAndNormal;

				pointAndNormalDict
				{
					point		(0 0 0.01);
					normal		(0 0 1);
					zone		porousZone;
				}
			}
			p13
			{
				type            cuttingPlane;
				planeType		pointAndNormal;

				pointAndNormalDict
				{
					point		(0 0 0.02);
					normal		(0 0 1);
					zone		porousZone;
				}
			}
			p14
			{
				type            cuttingPlane;
				planeType		pointAndNormal;

				pointAndNormalDict
				{
					point		(0 0 0.03);
					normal		(0 0 1);
					zone		porousZone;
				}
			}
			p15
			{
				type            cuttingPlane;
				planeType		pointAndNormal;

				pointAndNormalDict
				{
					point		(0 0 0.04);
					normal		(0 0 1);
					zone		porousZone;
				}
			}
			p16
			{
				type            cuttingPlane;
				planeType		pointAndNormal;

				pointAndNormalDict
				{
					point		(0 0 0.05);
					normal		(0 0 1);
					zone		porousZone;
				}
			}
			p17
			{
				type            cuttingPlane;
				planeType		pointAndNormal;

				pointAndNormalDict
				{
					point		(0 0 0.06);
					normal		(0 0 1);
					zone		porousZone;
				}
			}
			p18
			{
				type            cuttingPlane;
				planeType		pointAndNormal;

				pointAndNormalDict
				{
					point		(0 0 0.07);
					normal		(0 0 1);
					zone		porousZone;
				}
			}
			p19
			{
				type            cuttingPlane;
				planeType		pointAndNormal;

				pointAndNormalDict
				{
					point		(0 0 0.08);
					normal		(0 0 1);
					zone		porousZone;
				}
			}
			p20
			{
				type            cuttingPlane;
				planeType		pointAndNormal;

				pointAndNormalDict
				{
					point		(0 0 0.09);
					normal		(0 0 1);
					zone		porousZone;
				}
			}
			p21
			{
				type            cuttingPlane;
				planeType		pointAndNormal;

				pointAndNormalDict
				{
					point		(0 0 0.10);
					normal		(0 0 1);
					zone		porousZone;
				}
			}
		);

		fields
	    (
	        UMean
			pMean
	        turbulenceProperties:RMean
	        vorticityMean
	        lambVectorMean
		);
	}
	
	runTimeControl1
	{
		type            runTimeControl;
		libs            ("libutilityFunctionObjects.so");
		timeStart		350;
		writeControl	onEnd;
		conditions
		{
			tracer
			{
				type            minMax;
				functionObject  tracerVolAverage;
				fields          (volAverage(porousZone,tracer));
				value           0.05;
				mode			minimum;
			}
		}
	}
	
	#includeFunc residuals
}

// ************************************************************************* //

\end{lstlisting}

\section{system/decomposeParDict}
\begin{lstlisting}
/*--------------------------------*- C++ -*----------------------------------*\
| =========                 |                                                 |
| \\      /  F ield         | OpenFOAM: The Open Source CFD Toolbox           |
|  \\    /   O peration     | Version:  v1912                                 |
|   \\  /    A nd           | Website:  www.openfoam.com                      |
|    \\/     M anipulation  |                                                 |
\*---------------------------------------------------------------------------*/
FoamFile
{
    version     2.0;
    format      ascii;
    class       dictionary;
    object      decomposeParDict;
}
// * * * * * * * * * * * * * * * * * * * * * * * * * * * * * * * * * * * * * //

numberOfSubdomains    48;

method          scotch;

scotchCoeffs
{
}

constraints
{
	// Keep owner and neighbour on same processor for faces in zones
	faces
    {
        type    preserveFaceZones;
        zones   (interface interfaceBottom interfaceMiddle interfaceTop);
        enabled true;
    }
}

// ************************************************************************* //

\end{lstlisting}

\section{system/fvSchemes}
\begin{lstlisting}
/*--------------------------------*- C++ -*----------------------------------*\
| =========                 |                                                 |
| \\      /  F ield         | OpenFOAM: The Open Source CFD Toolbox           |
|  \\    /   O peration     | Version:  v1912                                 |
|   \\  /    A nd           | Website:  www.openfoam.com                      |
|    \\/     M anipulation  |                                                 |
\*---------------------------------------------------------------------------*/

FoamFile
{
    version 2.0;
    format ascii;
    class dictionary;
    location system;
    object fvSchemes;
}
// * * * * * * * * * * * * * * * * * * * * * * * * * * * * * * * * * * * * * //

    ddtSchemes
    {
        default	backward;
    }

    gradSchemes
    {
        default Gauss linear;
    }

    divSchemes
	{
		default         none;
		div(phi,U)      Gauss LUST grad(U);
		div(phi,nuTilda) Gauss limitedLinear 0.1;
		div((nuEff*dev2(T(grad(U))))) Gauss linear;
		
		div(phi,tracer) Gauss limitedLinear01 1;
	}

    interpolationSchemes
    {
        default			linear;
    }

    laplacianSchemes
    {
        default         Gauss linear orthogonal;
    }

    snGradSchemes
    {
        default			orthogonal;
    }

	wallDist
    {
        method meshWave;
    }

    fluxRequired
    {
        default no;
        p ;
        Phi ;
    }

// ************************************************************************* //

\end{lstlisting}

\section{system/fvSolution}
\begin{lstlisting}
/*--------------------------------*- C++ -*----------------------------------*\
| =========                 |                                                 |
| \\      /  F ield         | OpenFOAM: The Open Source CFD Toolbox           |
|  \\    /   O peration     | Version:  v1912                                 |
|   \\  /    A nd           | Website:  www.openfoam.com                      |
|    \\/     M anipulation  |                                                 |
\*---------------------------------------------------------------------------*/

FoamFile
{
    version 2.0;
    format ascii;
    class dictionary;
    location system;
    object fvSolution;
}
// * * * * * * * * * * * * * * * * * * * * * * * * * * * * * * * * * * * * * //

	PIMPLE
	{
		nOuterCorrectors 3;
		nCorrectors     3;
		nNonOrthogonalCorrectors 0;
		pRefPoint (0.15 0.15 0.1);
		pRefValue 0;
		
		residualControl
		{
			"(p|U)"
			{
				tolerance		1e-04;
				relTol			0;
			}
		}
		
		relaxationFactors
		{
			fields
			{
				p				0.4;
				pFinal			1;
			}
	
			equations
			{
				U				0.7;
				UFinal			1;
				nuTilda			1;
				nuTildaFinal	1;            
			}
		}
	}

    solvers
    {
		p
		{
			solver          GAMG;
			smoother        GaussSeidel;
			tolerance       1e-04;
			relTol          0.01;
			minIter			1;
        	maxIter         200;
		}
	
		pFinal
		{
			$p;
			smoother        GaussSeidel;
			tolerance       1e-04;
			relTol          0.01;
		}
	
		U
		{
			solver          PBiCGStab;
			preconditioner  diagonal;
			tolerance       1e-04;
			relTol          0.01;
			minIter			1;
          	maxIter         100;
		}
	
		UFinal
		{
			$U;
			tolerance       1e-04;
			relTol          0.01;
		}
		
		tracer
		{
			solver			PBiCGStab;
			preconditioner	diagonal;
            tolerance		1e-04;
            relTol          0.01;
            minIter			1;
		}
		
		Phi
		{
			solver          GAMG;
			smoother        GaussSeidel;
			tolerance       1e-06;
			relTol          0.01;
          	maxIter         20;
		}
    }

    relaxationFactors
	{
		fields
		{
			p				0.4;
			pFinal			1;
		}
	
		equations
		{
			U				0.7;
			UFinal			1;
			nuTilda			1;
			nuTildaFinal	1;            
		}
	
	}
	
	potentialFlow
	{
		nNonOrthogonalCorrectors 10;
	}

// ************************************************************************* //

\end{lstlisting}

\section{system/setFieldsDict}
\begin{lstlisting}
/*--------------------------------*- C++ -*----------------------------------*\
| =========                 |                                                 |
| \\      /  F ield         | OpenFOAM: The Open Source CFD Toolbox           |
|  \\    /   O peration     | Version:  v1912                                 |
|   \\  /    A nd           | Website:  www.openfoam.com                      |
|    \\/     M anipulation  |                                                 |
\*---------------------------------------------------------------------------*/
FoamFile
{
    version     2.0;
    format      ascii;
    class       dictionary;
    object      setFieldsDict;
}
// * * * * * * * * * * * * * * * * * * * * * * * * * * * * * * * * * * * * * //

defaultFieldValues
(
    volScalarFieldValue tracer 0
);

regions
(
    // Setting values inside a box
    boxToCell
    {
        box     (0.25 0.30 0) (0.50 0.45 0.10);
        fieldValues
        (
            volScalarFieldValue tracer 1
        );
    }
);


// ************************************************************************* //

\end{lstlisting}

\section{system/topoSetDict}
\begin{lstlisting}
/*--------------------------------*- C++ -*----------------------------------*\
| =========                 |                                                 |
| \\      /  F ield         | OpenFOAM: The Open Source CFD Toolbox           |
|  \\    /   O peration     | Version:  v1912                                 |
|   \\  /    A nd           | Website:  www.openfoam.com                      |
|    \\/     M anipulation  |                                                 |
\*---------------------------------------------------------------------------*/

FoamFile
{
    version     2.0;
    format      ascii;
    class       dictionary;
    object      topoSetDict;
}
// * * * * * * * * * * * * * * * * * * * * * * * * * * * * * * * * * * * * * //

actions
(
	{
		name    porousZone;
		type    cellZoneSet;
		action  new;
		source  boxToCell;
		sourceInfo
		{
			box (0.25 0.30 0) (0.50 0.45 0.1);
		}
	}
	
	{
		name    interfaceSelection;
		type    faceSet;
		action  new;
		source  boxToFace;
		sourceInfo
		{
			box (0.25 0.2999 0) (0.50 0.3001 0.1);
		}
	}
	
	{
		name     interfaceSelection;
		type     faceSet;
		action   subtract;
		source   normalToFace;
		normal   (0 1 0);
		cos      0.01;
	}
	
	{
		name     interfaceSelection;
		type     faceSet;
		action   subtract;
		source   normalToFace;
		normal   (0 0 1);
		cos      0.01;
	}
	
	{
		name	interface;
		type	faceZoneSet;
		action 	new;
		source	setToFaceZone;
		faceSet	interfaceSelection;
	}
	
	{
		name    interfaceBottom;
		type    faceSet;
		action  new;
		source  boxToFace;
		sourceInfo
		{
			box (0.25 0.2999 0) (0.50 0.3001 0.033);
		}
	}
	
	{
		name    interfaceMiddle;
		type    faceSet;
		action  new;
		source  boxToFace;
		sourceInfo
		{
			box (0.25 0.2999 0.033) (0.50 0.3001 0.066);
		}
	}
	
	{
		name    interfaceTop;
		type    faceSet;
		action  new;
		source  boxToFace;
		sourceInfo
		{
			box (0.25 0.2999 0.066) (0.50 0.3001 0.1);
		}
	}
	
	{
		name     interfaceBottom;
		type     faceSet;
		action   subtract;
		source   normalToFace;
		normal   (0 1 0);
		cos      0.01;
	}
	
	{
		name     interfaceBottom;
		type     faceSet;
		action   subtract;
		source   normalToFace;
		normal   (0 0 1);
		cos      0.01;
	}
	
	{
		name     interfaceMiddle;
		type     faceSet;
		action   subtract;
		source   normalToFace;
		normal   (0 1 0);
		cos      0.01;
	}
	
	{
		name     interfaceMiddle;
		type     faceSet;
		action   subtract;
		source   normalToFace;
		normal   (0 0 1);
		cos      0.01;
	}
	
	{
		name     interfaceTop;
		type     faceSet;
		action   subtract;
		source   normalToFace;
		normal   (0 1 0);
		cos      0.01;
	}
	
	{
		name     interfaceTop;
		type     faceSet;
		action   subtract;
		source   normalToFace;
		normal   (0 0 1);
		cos      0.01;
	}
	
	{
		name	interfaceBottom;
		type	faceZoneSet;
		action 	new;
		source	setToFaceZone;
		faceSet	interfaceBottom;
	}
	
	{
		name	interfaceMiddle;
		type	faceZoneSet;
		action 	new;
		source	setToFaceZone;
		faceSet	interfaceMiddle;
	}
	
	{
		name	interfaceTop;
		type	faceZoneSet;
		action 	new;
		source	setToFaceZone;
		faceSet	interfaceTop;
	}
);

\end{lstlisting}

\section{system/totalTKE}
\begin{lstlisting}
/*--------------------------------*- C++ -*----------------------------------*\
| =========                 |                                                 |
| \\      /  F ield         | OpenFOAM: The Open Source CFD Toolbox           |
|  \\    /   O peration     | Version:  v1912                                 |
|   \\  /    A nd           | Website:  www.openfoam.com                      |
|    \\/     M anipulation  |                                                 |
\*---------------------------------------------------------------------------*/
totalTKE
{
	type			coded;
	libs			("libutilityFunctionObjects.so");
	name			totalTKE;
	executeControl	timeStep;
	writeControl	writeTime;
	timeStart		155;
	// timeEnd		0;
	enabled			true;

/*---------------------------------------------------------------------------*\

	Total Turbulent Kinect Energy Evaluation
		** Requires fieldAverage Function to Obtain UPrime2Mean**
			** Resolved Reynolds Stress Tensor
		** Requires turbulenceFields Function to Obtain R**
			** Subgrid Reynolds Stress Tensor

\*---------------------------------------------------------------------------*/

	codeExecute
	#{
		static autoPtr<volScalarField> totalTKE;

		if
		(
			mesh().foundObject<volSymmTensorField>("UPrime2Mean")
			&&
			mesh().foundObject<volSymmTensorField>("turbulenceProperties:R")
			&&
			mesh().foundObject<volScalarField>("totalTKE") == 0
		)
		{
			Info << "Turbulent Kinect Energy:" << endl;
			Info << "	Initialising" << endl;
			Info << "	Calculating" << nl << endl;

			totalTKE.set
			(
				new volScalarField
				(
					IOobject
					(
						"totalTKE",
						mesh().time().timeName(),
						mesh(),
						IOobject::NO_READ,
						IOobject::AUTO_WRITE
					),
					mesh(),
					dimensionedScalar
					(
						"totalTKE",
						dimensionSet(0,2,-2,0,0,0,0),
						0
					)
				)
			);

			const volSymmTensorField& R = mesh().lookupObjectRef<volSymmTensorField>("turbulenceProperties:R");
			const volSymmTensorField& UPrime2Mean = mesh().lookupObjectRef<volSymmTensorField>("UPrime2Mean");

			volScalarField& totalTKE = mesh().lookupObjectRef<volScalarField>("totalTKE");
			totalTKE = (0.5 * tr(R)) + (0.5 * tr(UPrime2Mean));
		}

		else if
		(
			mesh().foundObject<volSymmTensorField>("UPrime2Mean")
			&&
			mesh().foundObject<volSymmTensorField>("turbulenceProperties:R")
			&&
			mesh().foundObject<volScalarField>("totalTKE")
		)
		{
			Info << "Turbulent Kinect Energy:" << endl;
			Info << "	Calculating" << nl << endl;

			const volSymmTensorField& R = mesh().lookupObjectRef<volSymmTensorField>("turbulenceProperties:R");
			const volSymmTensorField& UPrime2Mean = mesh().lookupObjectRef<volSymmTensorField>("UPrime2Mean");

			volScalarField& totalTKE = mesh().lookupObjectRef<volScalarField>("totalTKE");
			totalTKE = (0.5 * tr(R)) + (0.5 * tr(UPrime2Mean));
		}

		else
		{
			Info << "Turbulent Kinect Energy:" << endl;
			Warning << endl
					<< "	Unable to Calculate Turbulent Kinect Energy" << endl
					<< "	UPrime2Mean and/or R Unavailable" << endl
					<< "	Enable fieldAverage and turbulenceFields Functions" << nl << endl;
		}
	#};
}

\end{lstlisting}

\section{allClear}
\begin{lstlisting}[language=bash]
#!/bin/bash

# Saves 0.orig from being deleted
mv 0.orig foo

# Deletes Files
rm -r constant/polyMesh
rm -r processor*/
rm -r dynamicCode
rm -r log
rm -r 0.* [1-9]*

# Restores 0.orig
mv foo 0.orig

# Creates file for paraview
CASE=${PWD##*/}
touch $CASE.foam

\end{lstlisting}

\section{mesh}
\begin{lstlisting}[language=bash]
#!/bin/sh

case=${PWD##*/}

rm -rf log p* 0
mkdir log
cp -r 0.orig 0

{ # try
	echo -e "Compiled variables:\n"
    blockMesh > log/blockMesh.log &&
    printf '%*s' "${COLUMNS:-$(tput cols)}" '' | tr ' ' -
    echo -e "blockMesh completed without errors"
    #save your output

} || { # catch
    # save log for exceptio
    echo -e "An error occured on blockMesh"
    exit 1 
}
{
    topoSet >log/topoSet.log &&
    echo -e "topoSet completed without errors"
} || {
    echo -e "An error occured on topoSet"
    exit 1 
}
{
    checkMesh -allGeometry -allTopology -writeAllFields -writeSets vtk > log/checkMesh.log &&
    echo -e "checkMesh completed without errors"
} || {
    echo -e "An error occured on checkMesh"
    exit 1 
}

rm -rf dynamicCode

{
    setFields > log/setFields.log &&
    echo -e "setFields completed without errors"
} || {
    echo -e "An error occured on setFields"
    exit 1 
}

echo -e "Mesh constructed and checked."
echo -e "Tracer fields set."

\end{lstlisting}

\section{ramCache}
\begin{lstlisting}[language=bash]
#!/bin/bash

free && sync && echo 3 > /proc/sys/vm/drop_caches && free

\end{lstlisting}

\section{reconstructParParallel}
\begin{lstlisting}[language=bash]
#!/bin/bash
echo "
      K. Wardle 6/22/09, modified by H. Stadler Dec. 2013, minor fix Will Bateman Sep 2014.
      bash script to run reconstructPar in pseudo-parallel mode
      by breaking time directories into multiple ranges
     "
     
USAGE="
      USAGE: $0 -n <NP> -f fields -o <OUTPUTFILE>
        -f (fields) is optional, fields given in the form T,U,p; option is passed on to reconstructPar
  -t (times) is optional, times given in the form tstart,tstop
        -o (output) is optional 
"

#TODO: add flag to trigger deletion of original processorX directories after successful reconstruction
# At first check whether any flag is set at all, if not exit with error message
if [ $# == 0 ]; then
    echo "$USAGE"
    exit 1
fi

#Use getopts to pass the flags to variables
while getopts "f:n:o:t:" opt; do
  case $opt in
    f) if [ -n $OPTARG ]; then
  FIELDS=$(echo $OPTARG | sed 's/,/ /g')
  fi
      ;;
    n) if [ -n $OPTARG ]; then
  NJOBS=$OPTARG
  fi
      ;;
    o) if [ -n $OPTARG ]; then
  OUTPUTFILE=$OPTARG
       fi
      ;;
    t) if [ -n $OPTARG ]; then
  TLOW=$(echo $OPTARG | cut -d ',' -f1)
  THIGH=$(echo $OPTARG | cut -d ',' -f2)
  fi
      ;;
    \?)
      echo "$USAGE" >&2
      exit 1
      ;;
    :)
      echo "Option -$OPTARG requires an argument." >&2
      exit 1
      ;;
  esac
done

# check whether the number of jobs has been passed over, if not exit with error message
if [[ -z $NJOBS ]]
then
    echo "
      the flag -n <NP> is required!
       "
    echo "$USAGE"
    exit 1
fi

APPNAME="reconstructPar"

echo "running $APPNAME in pseudo-parallel mode on $NJOBS processors"

#count the number of time directories
NSTEPS=$(($(ls -d processor0/[0-9]*/ | wc -l)-1))
NINITAL=$(ls -d [0-9]*/ | wc -l) ##count time directories in case root dir, this will include 0

P=p
#find min and max time
TMIN=$(ls processor0 -1v | sed '/constant/d' | sort -g | sed -n 2$P) # modified to omit constant and first time directory
#TMIN=`ls processor0 | sort -nr | tail -1`
TMAX=$(ls processor0 -1v | sed '/constant/d' | sort -gr | head -1) # modified to omit constant directory
#TMAX=`ls processor0 | sort -nr | head -1`

#Adjust min and max time according to the parameters passed over
if [ -n "$TLOW" ]
  then
    TMIN=$(ls processor0 -1v | sed '/constant/d' | sort -g | sed -n 1$P) # now allow the first directory
    NLOW=2
    NHIGH=$NSTEPS
    # At first check whether the times are given are within the times in the directory
    if [ $(echo "$TLOW > $TMAX" | bc) == 1 ]; then
        echo "
      TSTART ($TLOW) > TMAX ($TMAX)
      Adjust times to be reconstructed!
      "
        echo "$USAGE"
        exit 1
    fi
    if [ $(echo "$THIGH < $TMIN" | bc) == 1 ]; then
        echo "
      TSTOP ($THIGH) < TMIN ($TMIN)
      Adjust times to be reconstructed!
      "
        echo "$USAGE"
        exit 1
    fi
  
    # Then set Min-Time
    until [ $(echo "$TMIN >= $TLOW" | bc) == 1 ]; do
      TMIN=$(ls processor0 -1v | sed -n $NLOW$P)
      NSTART=$(($NLOW))
      let NLOW=NLOW+1
    done

    # And then set Max-Time
    until [ $(echo "$TMAX <= $THIGH" | bc) == 1 ]; do
      TMAX=$(ls processor0 -1v | sed -n $NHIGH$P)
      let NHIGH=NHIGH-1
    done

    # Finally adjust the number of directories to be reconstructed
    NSTEPS=$(($NHIGH-$NLOW+3))

  else

    NSTART=2

fi

echo "reconstructing $NSTEPS time directories"

NCHUNK=$(($NSTEPS/$NJOBS))
NREST=$(($NSTEPS%$NJOBS))
TSTART=$TMIN

echo "making temp dir"
TEMPDIR="temp.parReconstructPar"
mkdir $TEMPDIR

PIDS=""
for i in $(seq $NJOBS)
do
  if [ $NREST -ge 1 ]
    then
      NSTOP=$(($NSTART+$NCHUNK))
      let NREST=$NREST-1
    else
      NSTOP=$(($NSTART+$NCHUNK-1))
  fi
  TSTOP=$(ls processor0 -1v | sed -n $NSTOP$P)


  if [ $i == $NJOBS ] 
  then
  TSTOP=$TMAX
  fi

  if [ $NSTOP -ge $NSTART ]
    then  
    echo "Starting Job $i - reconstructing time = $TSTART through $TSTOP"
    if [ -n "$FIELDS" ]
      then
        $($APPNAME -fields "($FIELDS)" -time $TSTART:$TSTOP > $TEMPDIR/output-$TSTOP &)
  echo "Job started with PID $(pgrep -n -x $APPNAME)"
  PIDS="$PIDS $(pgrep -n -x $APPNAME)" # get the PID of the latest (-n) job exactly matching (-x) $APPNAME
      else
        $($APPNAME -time $TSTART:$TSTOP > $TEMPDIR/output-$TSTOP &)
  echo "Job started with PID $(pgrep -n -x $APPNAME)"
  PIDS="$PIDS $(pgrep -n -x $APPNAME)"
    fi
   fi

  let NSTART=$NSTOP+1
  TSTART=$(ls processor0 -1v | sed -n $NSTART$P)
done

#sleep until jobs finish
#if number of jobs > NJOBS, hold loop until job finishes
NMORE_OLD=$(echo 0)
until [ $(ps -p $PIDS | wc -l) -eq 1 ]; # check for PIDS instead of $APPNAME because other instances might also be running 
  do 
    sleep 10
    NNOW=$(ls -d [0-9]*/ | wc -l) ##count time directories in case root dir, this will include 0
    NMORE=$(echo $NSTEPS-$NNOW+$NINITAL | bc) ##calculate number left to reconstruct and subtract 0 dir
    if [ $NMORE != $NMORE_OLD ]
      then
      echo "$NMORE directories remaining..."
    fi
    NMORE_OLD=$NMORE
  done

#combine and cleanup
if [ -n "$OUTPUTFILE" ] 
  then
#check if output file already exists
  if [ -e "$OUTPUTFILE" ] 
  then
    echo "output file $OUTPUTFILE exists, moving to $OUTPUTFILE.bak"
    mv $OUTPUTFILE $OUTPUTFILE.bak
  fi

  echo "cleaning up temp files"
  for i in $(ls $TEMPDIR)
  do
    cat $TEMPDIR/$i >> $OUTPUTFILE
  done
fi

rm -rf $TEMPDIR

echo "finished"


\end{lstlisting}
