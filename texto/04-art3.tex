\chapter{Velocity Estimatives in Vegetated Lateral Cavities}
\label{chap:art3}
In this chapter, the second topic of the dissertation is further developed in a published conference paper. The objective of this paper was to develop a simple numerical method capable of estimating the flow and the mass exchange between a lateral cavity and the main channel. Differently of the previous chapter, this paper introduces an open source approach to the problem, making the model further accessible to the general public. Also, the numerical model was further developed to account the mass transfer between the regions.

The original paper was published in the 'XIII Encontro Nacional de Águas Urbanas', on October 2020, Porto Alegre, Brazil.
\section*{Authors}
\begin{itemize}
    \item Luiz Eduardo Domingos de Oliveira \footnote{Federal University of Mato Grosso do Sul}
    \item Taís Natsumi Yamasaki \footnotemark[1]
    \item Johannes Gérson Janzen \footnotemark[1]
    \item Carlo Gualtieri \footnote{University of Naples Federico II}
\end{itemize}
\addcontentsline{toc}{section}{Abstract}
\section*{Abstract}
Lateral cavities are a type of transient storage zones that occur in riverine systems. They play an important role in mass transport processes, especially due to a higher residence time. In this study, a numerical simulation of flow past a lateral cavity with vegetation was performed to assess the impact of the vegetation on the cavity hydrodynamics. The vegetation drag was introduced in a simplified method, as it was modelled as an anisotropic porous medium. The model could reproduce the experimental results at a reduced computational cost and can be considered a study platform for future studies.

\noindent\textbf{Keywords:} Lateral Cavities; Vegetation; Computational Fluid Dynamics (CFD).

\section{Introduction}

\addcontentsline{toc}{section}{References}
\printbibliography[segment=\therefsegment,heading=subbibliography, title={References}]
