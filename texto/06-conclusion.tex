\chapter{Conclusion and Recommendations}
\label{chap:conclusion}

The objective of this study was the description of the hydrodynamics and mass exchange in dead waters. In the present study, numerical experiments were performed to describe the flow in groynes and lateral cavities.

First, a literature review of the flow conditions and methods used to describe the flow was performed which showed gaps in knowledge to be fulfilled. Second, the first description of the groyne flow was presented along with the study of tracer fields, which showed that the mass exchange between the DZ and the main channel is not unique and occurs at two different rates depending on the elapsed time. Third, the first description of lateral cavities was presented, in this paper, we described the flow using a model developed in commercial software. Forth, the description of lateral cavities was further developed using another approach to the turbulence fields and an open-source package. Finally, the effects of vegetation in lateral cavities were described.

The differences in modelling groynes and lateral cavities are significant. The study of groynes implies a periodicity that must be fulfilled by two means: a) a series of groyne fields or b) a pair of cyclic/periodic surfaces. The difficulties inhered by both methods rely on the high computational cost, as in the option a) the domain is extensive and the meshing process is harder and usually means in a loss of detail as a refined mesh becomes prohibitive. On another hand, the second option provides a more accurate description of the flow as the meshing can be concentrated only on one groyne field, although the implementation of a periodic surface in a zone of high mixture implies in a requirement for a small cell size especially in the groyne head, a region of intense vortex shredding. In contrast to groynes, cavities do not require repetition and can be represented in a simple inlet/outlet scheme, this implies reduced computational costs.

For both studies, as turbulence is the main phenomena in DZ the selection of the turbulence model is primordial to the model. Through the investigation process, it was noticed that the Reynolds Averaging Navier-Stokes can only give an approximation of the flow, we analysed multiple models that rely on this spectra and only the k-$\omega$ SST model was suitable for this kind of flow. Another hybrid model such as the Detached Eddy Simulation was a further improvement to the description of turbulence. Although, some structures were clearer when the Large Eddy Simulation was introduced, as the instantaneous flow was considered.

For vegetated flows, the approximation using porous media to represent the vegetation drag was used. The results of this model proved that this approach is viable for DZ. The effects of the vegetation density in the hydrodynamics and mass exchange proved to follow different phases. The lateral cavity can present a structure that was not anticipated for the given $W/L$ region as a secondary circulation appears when $a>5.3280$\%. This secondary circulation changes how mass is exchanged, similar to the first paper of this dissertation, the exchange values are changed due to the concentration of mass in the secondary gyre that has no contact with the main channel.

This dissertation enriched the knowledge of dead zones vegetated/non-vegetated. It showed different modelling techniques for two types of DZ: lateral cavity and groyne fields. Furthermore, it shows that vegetation can drastically alter the flow by reducing the velocity, TKE and vorticity, this influence could promote the deposition of fine sediments and organic matter. Additionally, it shows that the vegetation can cause a threshold in the mass exchange between the main channel and the lateral cavity, in which the rate is drastically reduced due to high blockage effects. This knowledge could help river managers to set limits and adjust the vegetation density inside the cavity to keep the desirable ecological function of the cavity.
