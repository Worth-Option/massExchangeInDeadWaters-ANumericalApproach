\noindent
In this appendix the script used to determine the grid convergence index is presented. This code is an automated script based on \textcite{celik2008} and \textcite{Dutta2018}.
\section{File Structure}
\noindent
The file structure of the script is shown bellow:
\\
\dirtree{%
.1 /.
.2 bin.
.3 import.py\DTcomment{CSV import script}.
.3 gci.py\DTcomment{Data analysis script for RANS calculations}.
.3 gciLES.py\DTcomment{Data analysis script for LES calculations}.
.3 plot.py\DTcomment{Plotting and export script}.
.2 treatment\DTcomment{User Created Directory}.
.3 results\DTcomment{Software Created Directory}.
.2 main.py.
}
\noindent
The requirements of the script are:
\begin{itemize}
\item Python 3.x
\item Scipy
\item Numpy
\item Pandas
\item Matplotlib
\end{itemize}
\section{main.py}
The user must provide a folder named treatment where three different \textit{csv} files must be placed. The execution of the code depends only on the main.py file that must be run in a python terminal:

\begin{lstlisting}[language=python]
#!/usr/bin/env python3
# -*- coding: utf-8 -*-
#
#  main.py
#  
#  Copyright 2020 Luiz Oliveira
#  
#  This program is free software; you can redistribute it and/or modify
#  it under the terms of the GNU General Public License as published by
#  the Free Software Foundation; either version 2 of the License, or
#  (at your option) any later version.
#  
#  This program is distributed in the hope that it will be useful,
#  but WITHOUT ANY WARRANTY; without even the implied warranty of
#  MERCHANTABILITY or FITNESS FOR A PARTICULAR PURPOSE.  See the
#  GNU General Public License for more details.
#  
#  You should have received a copy of the GNU General Public License
#  along with this program; if not, write to the Free Software
#  Foundation, Inc., 51 Franklin Street, Fifth Floor, Boston,
#  MA 02110-1301, USA.
#  
#  

"""
Main module

This script analyses numerical and modeling errors in LES simulations and
Grid convergence analysis on RANS simulations.
The analysis steps are performed by the modules in the bin folder
"""

import sys
import os
import shutil
import time
start_time = time.time()

def cls():
    """
    Clears the prompt
    """
    os.system('cls' if os.name=='nt' else 'clear')

# Check for necessary directories
if not os.path.exists('treatment'):
    os.makedirs('treatment')
    print("The directory treatment/ was created, please populate with the "
          "desired csv files to be analysed.")
    sys.exit('The directory treatment/ did not exist.')
elif not os.listdir('treatment'):
    sys.exit('The directory treatment/ is empty.')
    
# Clear the previous results directories
if os.path.exists('treatment/results'):
    shutil.rmtree('treatment/results')
os.makedirs('treatment/results')

analysisType = input("Type of Analysis\n[1] RANS\n[2] LES\nChosen Option: ")

if analysisType == 1:
    # Import CSV
    exec(open("bin/import.py").read());
    # Grid Convergence Analysis (RANS)
    exec(open("bin/gci.py").read());
else:
    # Grid Convergence Analysis (LES)
    exec(open("bin/gciLES.py").read());

\end{lstlisting}
\section{import.py}
The import process occurs in bin/import.py file:
\begin{lstlisting}[language=python]
#!/usr/bin/env python3
# -*- coding: utf-8 -*-

# Libraries
import os
import re
import pandas as pd
from detect_delimiter import detect

def cls():
    """
    Clears the prompt
    """
    os.system('cls' if os.name=='nt' else 'clear')

# Input delimeter and file names
coarserFile = input("Name of coarser mesh file: ")
mediumFile = input("Name of medium mesh file: ")
finerFile = input("Name of finer mesh file: ")

coarserFile = "treatment/" + coarserFile
mediumFile = "treatment/" + mediumFile
finerFile = "treatment/" + finerFile

with open(coarserFile) as f:
    for line in f:
        if re.match(r"^\d+.*$",line):
            delim = detect(line)
            break
if delim is None:
    delim = input("""Type of delimiter\n[1] ('\\t')\n[2] (' ')\n[3] (';')
                  [4] (',')\n [5] Custom delimiter\nChosen option: """)
    delim = int(delim)
    if delim == 1:
        delim = '\\t'
    elif delim == 2:
        delim = ' '
    elif delim == 3:
        delim = ';'
    elif delim == 4:
        delim = ','
    elif delim == 5:
        delim = input("Enter custom delimiter: ")
        
cls()

print("Python columns start on zero, please pay attention to this detail.\n")
axis = int(input("Axis column number: "))
var = int(input("Variable column number: "))

headerlines = int(input("Number of header lines: "))

# Import generated data
coarser = pd.read_csv(coarserFile,delimiter=delim, skiprows=headerlines,
                      usecols=[axis,var], header=0,
                      names=["Axis","Variable_coarser"])
medium = pd.read_csv(mediumFile,delimiter=delim, skiprows=headerlines,
                      usecols=[axis,var], header=0,
                      names=["Axis","Variable_medium"])
finer = pd.read_csv(finerFile,delimiter=delim, skiprows=headerlines,
                      usecols=[axis,var], header=0,
                      names=["Axis","Variable_finer"])

# Reindexing using axis
coarser = coarser.set_index('Axis')
medium = medium.set_index('Axis')
finer = finer.set_index('Axis')

# Sorting imported data
coarser = coarser.sort_values('Axis')
medium = medium.sort_values('Axis')
finer = finer.sort_values('Axis')

cls()

\end{lstlisting}
\section{gci.py}
The processing occurs in bin/gci.py file for RANS calculations:

\begin{lstlisting}[language=python]
#!/usr/bin/env python3
# -*- coding: utf-8 -*-

import os
import pandas as pd
import numpy as np

def cls():
    """
    Clears the prompt
    """
    os.system('cls' if os.name=='nt' else 'clear')

cElements = int(input("Number of elements of the coarser mesh: "))
mElements = int(input("Number of elements of the medium mesh: "))
fElements = int(input("Number of elements of the finer mesh: "))

analysisType = input("Type of Analysis\n[1] 2D\n[2] 3D\nChosen Option: ")
volume = float(input("Total cell volume [m3]: "))

if analysisType == '1':
	h1 = (volume/fElements)**(0.5)
	h2 = (volume/mElements)**(0.5)
	h3 = (volume/cElements)**(0.5)
	
elif analysisType == '2':
	h1 = (volume/fElements)**(1/3)
	h2 = (volume/mElements)**(1/3)
	h3 = (volume/cElements)**(1/3)
	
else:
	cls()
	print("Deleting all data...")
	print("Computer shutting down...")

# Refinement rate

r21 = h2/h1
r32 = h3/h2

# Variable absolute error
desiredVar = pd.concat([finer, medium, coarser], axis=1)
desiredVar = desiredVar.interpolate('index').reindex(medium.index)
e21 = desiredVar.Variable_medium - desiredVar.Variable_finer
e32 = desiredVar.Variable_coarser - desiredVar.Variable_medium
desiredVar['e21'] = e21
desiredVar['e32'] = e32

# Sign
sign = np.sign(desiredVar['e32']/desiredVar['e21'])
desiredVar['Sign'] = sign.astype(float)

# Order Error
initial = np.repeat(2.0, len(desiredVar.index))

def aparentOrder(order, df):
    order = np.abs(order)
    q = np.log(((r21**order)-desiredVar.Sign)/((r32**order)-desiredVar.Sign))
    ap = np.abs(np.log(np.abs(desiredVar['e32']/desiredVar['e21'])+q))/np.log(r21)
    error = np.abs(order - ap)
    error = np.array(error.values.tolist()) #converts to array
    return np.mean(error)

res = optimize.minimize(aparentOrder, args=(desiredVar),
                        x0=initial, method = 'Nelder-Mead', tol=0.01,
                        options={'maxiter':1000})

order = res.x
q = np.log((r21**order-desiredVar.Sign)/(r32**order-desiredVar.Sign))
ap = np.abs(np.log(np.abs(desiredVar['e32']/desiredVar['e21'])+q))/np.log(r21)
orderError = order - ap

desiredVar['Aparent Order'] = ap
desiredVar['Optimized Order'] = order
desiredVar['Order Error'] = orderError

# Extrapolated values
ext21 = ((r21**ap)*desiredVar.Variable_finer-desiredVar.Variable_medium)/((r21**ap)-1)
ext32 = ((r32**ap)*desiredVar.Variable_medium-desiredVar.Variable_coarser)/((r32**ap)-1)

desiredVar['Extrapolated Value (Finer, Medium)'] = ext21
desiredVar['Extrapolated Value (Medium, Coarser)'] = ext32

# Calculate and report the error estimatives
apxRelErr = np.abs((desiredVar.Variable_finer-desiredVar.Variable_medium)/desiredVar.Variable_finer)
extRelErr = np.abs((ext21-desiredVar.Variable_finer)/ext21)
gci = (1.25*apxRelErr)/((r21**ap)-1)

desiredVar['Aproximated Relative Error'] = apxRelErr
desiredVar['Extrapolated Relative Error'] = extRelErr
desiredVar['Grid Convergence Index'] = gci

# Export generated table
desiredVar.to_excel("treatment/results/gci.xlsx")   

\end{lstlisting}

\section{gciLES.py}
The processing occurs in bin/gciLES.py file for LES calculations:
\begin{lstlisting}[language=python]
#!/usr/bin/env python3
# -*- coding: utf-8 -*-
#
#  gciLES.py
#  
#  Copyright 2020 Luiz Oliveira
#  
#  This program is free software; you can redistribute it and/or modify
#  it under the terms of the GNU General Public License as published by
#  the Free Software Foundation; either version 2 of the License, or
#  (at your option) any later version.
#  
#  This program is distributed in the hope that it will be useful,
#  but WITHOUT ANY WARRANTY; without even the implied warranty of
#  MERCHANTABILITY or FITNESS FOR A PARTICULAR PURPOSE.  See the
#  GNU General Public License for more details.
#  
#  You should have received a copy of the GNU General Public License
#  along with this program; if not, write to the Free Software
#  Foundation, Inc., 51 Franklin Street, Fifth Floor, Boston,
#  MA 02110-1301, USA.
#  
#  

"""
Main module

This script analyses numerical and modeling errors in LES simulations
"""

import os
import re
import openpyxl
import pandas as pd
import matplotlib.pyplot as plt
from detect_delimiter import detect
from scipy.optimize import fsolve

def cls():
    """
    Clears the prompt
    """
    os.system('cls' if os.name=='nt' else 'clear')

def caseInfo(ncases):
    """
    Reads the case information for an n number of simulations
    """
    d = {'Elements' : [], 'DeltaT' : [], 'Volume' : []}
    for ii in range(ncases):
        elmt = int(input("Number of elements of the mesh {0}: ".format(ii)))
        dt = float(input("Timestep size of the mesh {0}: ".format(ii)))
        d['Elements'].append(elmt)
        d['DeltaT'].append(dt)
    d['Volume'].append(float(input("Total cell volume [m3]: ")))
    df = pd.DataFrame(dict([ (k,pd.Series(v)) for k,v in d.items() ]))
    df.index.names = ['Mesh']
    df.to_csv('treatment/caseInformation.csv', index = False)
    return df

def checkDelimiter(filename, directory):
    """
    Checks the delimiter of a file or takes the input from the user
    """
    with open(os.path.join(directory,filename)) as f:
        for line in f:
            if re.match(r"^\d+.*$",line):
                delim = detect(line)
                break
    if delim is None:
        delim = input("""Type of delimiter\n[1] ('\\t')\n[2] (' ')\n[3] (';')
                      [4] (',')\n [5] Custom delimiter\nChosen option: """)
        delim = int(delim)
        if delim == 1:
            delim = '\\t'
        elif delim == 2:
            delim = ' '
        elif delim == 3:
            delim = ';'
        elif delim == 4:
            delim = ','
        elif delim == 5:
            delim = input("Enter custom delimiter: ")
    cls()
    return delim

def caseImport(ncases):
    """
    Imports the data from an n number of simulations
    """
    directory = 'treatment'
    # Get all files.
    list = os.listdir(directory)
    filetpl = []
    for file in list:
        # Use join to get full file path.
        location = os.path.join(directory, file)
        # Get size and add to list of tuples.
        size = os.path.getsize(location)
        filetpl.append((size, file))
    # Sort list of tuples by the first element, size.
    filetpl.sort(key=lambda s: s[0])
    filetpl.reverse()
    df = pd.DataFrame(data=filetpl, columns=["Size","Filename"])
    df.drop(df.tail(len(list)-ncases).index, inplace = True)
    df.index.names = ['Mesh']
    print("Assuming this file order:")
    print(df.to_string())
    order = int(input("Is this corrrect?\n[1] Yes\n[2] No\nChoice: "))
    if order == 2:      
        lst = []
        print("Write the mesh number succeeded by the file name:\n")
        for ii in range(ncases): 
            file = [int(input()), input()] 
            lst.append(file) 
        df = pd.DataFrame(data=lst, columns=["Size","Filename"])
        df.index.names = ['Mesh']
        cls()
        print("Files to be imported:\n")
        print(df.to_string())
    importedFiles = dict()
    delim = checkDelimiter(df.Filename[0],directory)
    for ii in range(ncases):
        temp = pd.read_csv(os.path.join(directory,df.Filename[ii]),
                           delimiter=delim)
        importedFiles['Mesh '+str(ii)] = temp
    return importedFiles

def refinementRate(df):
    """
    Defines the refinement rate between the meshes
    """
    r = list()
    for n in range(testVersion):
        if n == testVersion - 1:
            refRate = 1
        else:
            refRate = df.Elements[n+1]/df.Elements[n]
        r.append(refRate)
    df['r'] = r
    return df

# Import the case structure data (mesh and timestep)
testVersion = int(input("""Which test should be performed?
[1] Short Version (3 cases)
[2] Long Version (5 cases)
Choice: """))
if testVersion == 1:
    testVersion = 3
else:
    testVersion = 5

infoFile = 'treatment/caseInformation.csv'
if os.path.exists(infoFile):
    infoDf = pd.read_csv(infoFile)
else: 
    print ("Please state the meshes from the finer to the coarser")  
    infoDf = caseInfo(testVersion)

# Import simulation data
nVar = int(input("""[1] Single data point
[2] Multiple data point (line)
Choice: """))
cls()
var = input("Write the name of the desired variable: ")
axis = input("Write the name of the desired plot axis: ")
if nVar == 1:
    cls()
    print("Please insert the point value for the meshes")
    simDf = dict()
    for ii in range(testVersion):
        jj = str(ii)
        lst = [float(input("Mesh "+jj+" value: "))]
        simDf['Mesh '+jj] = pd.DataFrame(data=lst,columns=[var])
    del ii,jj,lst
elif nVar == 2:
    simDf = caseImport(testVersion)
del nVar

# Starts evaluating the GCI
infoDf.eval('h = (@infoDf.Volume[0]/Elements)**(1/3)', inplace=True)
infoDf.eval('hstar = (h*DeltaT)**(1/2)', inplace=True)
infoDf = refinementRate(infoDf)
delta = max(infoDf.h)
r = infoDf.r.mean()
hstar = infoDf.hstar.mean()

## Simulated data
s1 = simDf['Mesh 0'][var]
s2 = simDf['Mesh 1'][var]
s3 = simDf['Mesh 2'][var]
if testVersion == 5:
    s4 = simDf['Mesh 3'][var]
    s5 = simDf['Mesh 4'][var]

if testVersion == 3:
    # Simplified method

    pn = 1.7
    pm = 1.5
    cm = (r**(1.7)*(s1-s2)-(s2-s3))/((r**(1.7)-r**(1.5)-r**(3.2)+r**(3))\
          *delta**(1.5))
    sc = ((r**(1.7)*s1-s2)*(r**(3.2)-r**(3))-(r**(1.7)*s2-s3)*(r**(1.7)\
          -r**(1.5)))/((r**(1.7)-1)*((r**(3.2)-r**(3))-(r**(1.7)-r**(1.5))))
    cn = (s1-sc-cm*delta**(1.5))/(hstar**(1.7))
    
    Enum = dict()
    Enum[0] = cn*(hstar**1.7)
    Enum[1] = cn*(r**1.7)*(hstar**1.7)
    Enum[2] = cn*(r**3.4)*(hstar**1.7)
    
    Emod = dict()
    Emod[0] = cm*(delta**1.5)
    Emod[1] = cm*(r**1.5)*(delta**1.5)
    Emod[2] = cm*(r**3)*(delta**1.5)
    
    jj=0
    for ii in simDf:
        simDf[ii]['Sc'] = sc
        simDf[ii]['Numerical Error'] = Enum[jj]
        simDf[ii]['Modelling Error'] = Emod[jj]
        simDf[ii]['Total Error'] = Enum[jj] + Emod[jj]
        jj+=1
    del ii,jj

elif testVersion == 5:
    # Full method

    def fullMethod(vars):
    # =========================================================================
    #     Sets the nonlinear system of 5 equations
    # =========================================================================
        sc, cn, cm, pn, pm = vars
        eq1 = cn*hstar**pn + cm*delta**pm
        eq2 = cn*(r*hstar)**pn + cm*(r*delta)**pm
        eq3 = cn*((r**2)*hstar)**pn + cm*((r**2)*delta)**pm
        eq4 = cn*((r**3)*hstar)**pn + cm*((r**3)*delta)**pm
        eq5 = cn*((r**4)*hstar)**pn + cm*((r**4)*delta)**pm
        return [eq1, eq2, eq3, eq4, eq5]
    
    sc, cn, cm, pn, pm =  fsolve(fullMethod, (0.007, 1, 1, 1.7, 1.5))
    Enum = dict()
    for ii in range(testVersion):
        if ii == 0:
            val = cn*hstar**pn
        else:
            val = cn*((r**ii)*hstar)**pn
        Enum[ii]=val
    
    Emod = dict()
    for ii in range(testVersion):
        if ii == 0:
            val = cm*delta**pm
        else:
            val = cm*((r**ii)*delta)**pm
        Emod[ii]=val
    
    jj=0
    for ii in simDf:
        simDf[ii]['Sc'] = sc
        simDf[ii]['Numerical Error'] = Enum[jj]
        simDf[ii]['Modelling Error'] = Emod[jj]
        simDf[ii]['Total Error'] = Enum[jj] + Emod[jj]
        jj+=1
    del ii, jj, val, var

# Export Results to Excel
d = {'Order of Accuracy for the Numerical Error (Pn)': pn,
     'Order of Accuracy for the Modelled Error (Pm)': pm,
     'Mean Constant for Numerical Errors (Cn)': cn.mean(),
     'Mean Constant for Modelled Errors (Cm)': cm.mean(),
     'Delta': delta,
     'Hstar': hstar,
     'Mean Refinement Rate': r
    }
idx = [0]

summary = pd.DataFrame(data=d, index=idx)
xlsxFile = 'treatment/results/dataSummary.xlsx'
if not os.path.isfile(xlsxFile):
    wb = openpyxl.Workbook()
    wb.save(xlsxFile)
    
with pd.ExcelWriter(xlsxFile, engine="openpyxl", mode='a') as writer:
    summary.to_excel(writer, sheet_name='Summary', index=False)
    for df_name, df in simDf.items():
        df.to_excel(writer, sheet_name=df_name, index=False)

del d, idx, xlsxFile

# Plot with error bars
fig, ax = plt.subplots(figsize=(9,6), dpi=300)
ax.plot(simDf['Mesh 0'][axis], simDf['Mesh 0'][var],
            label= 'Mesh 0', aa=True)
ax.plot(simDf['Mesh 1'][axis], simDf['Mesh 1'][var],
            label= 'Mesh 1', aa=True)
ax.plot(simDf['Mesh 2'][axis], simDf['Mesh 2'][var],
            label= 'Mesh 2', aa=True)
if testVersion == 5:
    ax.plot(simDf['Mesh 3'][axis], simDf['Mesh 3'][var],
            label= 'Mesh 3 - Coarser', aa=True)
    ax.plot(simDf['Mesh 4'][axis], simDf['Mesh 4'][var],
            label= 'Mesh 4 - Coarser', aa=True)

ax.legend(loc='best',fontsize='x-large')

plt.grid()
plt.autoscale(enable=True, tight=True)
plt.xlabel(axis,fontsize='x-large')
plt.ylabel(var,fontsize='x-large')
plt.savefig('treatment/results/allMeshes.png', bbox_inches='tight')

fig, ax = plt.subplots(figsize=(9,6), dpi=300)
l, caps, c = plt.errorbar(simDf['Mesh 1'][axis], simDf['Mesh 1'][var],
            simDf['Mesh 1']['Total Error'],
            elinewidth = 1, capsize = 5, capthick = 1, marker = 'o',
#            errorevery = 5,
            uplims = True, lolims = True, 
            lw=1.5, aa = True)

for cap in caps:
    cap.set_marker("_")
    
plt.grid()
plt.autoscale(enable=True, tight=True)
plt.xlabel(axis,fontsize='x-large')
plt.ylabel(var,fontsize='x-large')
plt.savefig('treatment/results/Mesh1.png', bbox_inches='tight')

\end{lstlisting}
\section{plot.py}
And finally the ploting and the spreadsheet containing the results are output by bin/plot.py file: 
\begin{lstlisting}[language=python]
#!/usr/bin/env python3
# -*- coding: utf-8 -*-

import matplotlib.pyplot as plt

fig, ax = plt.subplots(figsize=(9,6), dpi=300)
ax.plot(desiredVar.index, desiredVar.Variable_coarser,
            label= 'Coarser', aa=True)
ax.plot(desiredVar.index, desiredVar.Variable_medium,
            label= 'Medium', aa=True)
ax.plot(desiredVar.index, desiredVar.Variable_finer,
            label= 'Finer', aa=True)

ax.legend(loc='best',fontsize='x-large')

plt.grid()
plt.autoscale(enable=True, tight=True)
plt.savefig('treatment/results/allMeshes.png')

fig, ax = plt.subplots(figsize=(9,6), dpi=300)
ax.errorbar(desiredVar.index, desiredVar.Variable_medium,
            gci*desiredVar.Variable_medium,
            errorevery = 5, elinewidth = 1,
            uplims = True, lolims = True, 
            lw=1.5, aa = True)

plt.grid()
plt.autoscale(enable=True, tight=True)
plt.savefig('treatment/results/mediumWithErrorbars.png')

fig, ax = plt.subplots(figsize=(9,6), dpi=300)
ax.errorbar(desiredVar.index, desiredVar.Variable_finer,
            gci*desiredVar.Variable_finer,
            errorevery = 5, elinewidth = 1,
            uplims = True, lolims = True, 
            lw=1.5, aa = True)

plt.grid()
plt.autoscale(enable=True, tight=True)
plt.savefig('treatment/results/finerWithErrorbars.png')
\end{lstlisting}