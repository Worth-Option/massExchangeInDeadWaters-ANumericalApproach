\documentclass[../main.tex]{subfiles}

\begin{document}
\noindent
In this appendix the script used to determine the grid convergence index is presented. This code is an automated script based on \textcite{celik2008}.
\section{File Structure}
\noindent
The file structure of the script is shown bellow:
\\
\dirtree{%
.1 /.
.2 bin.
.3 import.py\DTcomment{CSV import script}.
.3 gci.py\DTcomment{Data analysis script}.
.3 plot.py\DTcomment{Plotting and export script}.
.2 treatment\DTcomment{User Created Directory}.
.3 results\DTcomment{Software Created Directory}.
.2 main.py.
}
\noindent
The requirements of the script are:
\begin{itemize}
\item Python 3.x
\item Scipy
\item Numpy
\item Pandas
\item Matplotlib
\end{itemize}
\section{main.py}
The user must provide a folder named treatment where three different \textit{csv} files must be placed. The execution of the code depends only on the main.py file that must be run in a python terminal:

\begin{lstlisting}[language=python]
#!/usr/bin/env python3
# -*- coding: utf-8 -*-

import os

def cls():
    os.system('cls' if os.name=='nt' else 'clear')

# Create Results Folder
if not os.path.exists('treatment/results'):
    os.makedirs('treatment/results')

# Import CSV
exec(open("bin/import.py").read());
cls()
print('Variables imported...')

# Grid Convergence Analysis
exec(open("bin/gci.py").read());
cls()
print('Grid convergence analysis performed...')

# Plot data with error bars
exec(open("bin/plot.py").read());
cls()
print('All done.')
\end{lstlisting}
\section{import.py}
The import process occurs in bin/import.py file:
\begin{lstlisting}[language=python]
#!/usr/bin/env python3
# -*- coding: utf-8 -*-

# Libraries
import pandas as pd

# Input delimeter and file names
coarserFile = input("Name of coarser mesh file: ")
mediumFile = input("Name of medium mesh file: ")
finerFile = input("Name of finer mesh file: ")

coarserFile = "treatment/" + coarserFile
mediumFile = "treatment/" + mediumFile
finerFile = "treatment/" + finerFile

delim = input("Type of delimiter\n[1] ('\\t')\n[2] (' ')\n[3] (';')\n[4] (',')\nChosen option: ")
delim = int(delim)
if delim == 1:
    delim = '\\t'
elif delim == 2:
    delim = ' '
elif delim == 3:
    delim = ';'
elif delim == 4:
    delim = ','

axis = int(input("Axis column number: "))
var = int(input("Variable column number: "))

headerlines = int(input("Number of header lines: "))

# Import generated data
coarser = pd.read_csv(coarserFile,delimiter=delim, skiprows=headerlines,
                      usecols=[axis,var], header=0,
                      names=["Axis","Variable_coarser"])
medium = pd.read_csv(mediumFile,delimiter=delim, skiprows=headerlines,
                      usecols=[axis,var], header=0,
                      names=["Axis","Variable_medium"])
finer = pd.read_csv(finerFile,delimiter=delim, skiprows=headerlines,
                      usecols=[axis,var], header=0,
                      names=["Axis","Variable_finer"])

# Reindexing using axis
coarser = coarser.set_index('Axis')
medium = medium.set_index('Axis')
finer = finer.set_index('Axis')

# Sorting imported data
coarser = coarser.sort_values('Axis')
medium = medium.sort_values('Axis')
finer = finer.sort_values('Axis')
\end{lstlisting}
\section{gci.py}
The processing occurs in bin/gci.py file:

\begin{lstlisting}[language=python]
#!/usr/bin/env python3
# -*- coding: utf-8 -*-

import pandas as pd
import numpy as np
from scipy import optimize

cElements = int(input("Number of elements of the coarser mesh: "))
mElements = int(input("Number of elements of the medium mesh: "))
fElements = int(input("Number of elements of the finer mesh: "))

analysisType = input("Type of Analysis\n[1] 2D\n[2] 3D\nChosen Option: ")
volume = float(input("Total cell volume [m3]: "))

if analysisType == '1':
	h1 = (volume/fElements)**(0.5)
	h2 = (volume/mElements)**(0.5)
	h3 = (volume/cElements)**(0.5)
	
elif analysisType == '2':
	h1 = (volume/fElements)**(1/3)
	h2 = (volume/mElements)**(1/3)
	h3 = (volume/cElements)**(1/3)
	
else:
	cls()
	print("Deleting all data...")
	print("Computer shutting down...")

# Refinement rate

r21 = h2/h1
r32 = h3/h2

# Variable absolute error
desiredVar = pd.concat([finer, medium, coarser], axis=1)
desiredVar = desiredVar.interpolate('index').reindex(medium.index)
e21 = desiredVar.Variable_medium - desiredVar.Variable_finer
e32 = desiredVar.Variable_coarser - desiredVar.Variable_medium
desiredVar['e21'] = e21
desiredVar['e32'] = e32
desiredVar = desiredVar.dropna()

# Sign
sign = np.sign(desiredVar['e32']/desiredVar['e21'])
desiredVar['Sign'] = sign.astype(float)

# Order Error
initial = np.repeat(2.0, len(desiredVar.index))

def aparentOrder(order, df):
    order = np.abs(order)
    q = np.log(((r21**order)-desiredVar.Sign)/((r32**order)-desiredVar.Sign))
    ap = np.abs(np.log(np.abs(desiredVar['e32']/desiredVar['e21'])+q))/np.log(r21)
    error = np.abs(order - ap)
    error = np.array(error.values.tolist()) #converts to array
    return np.mean(error)

res = optimize.minimize(aparentOrder, args=(desiredVar),
                        x0=initial, method = 'Nelder-Mead', tol=0.01,
                        options={'maxiter':1000})

order = res.x
q = np.log((r21**order-desiredVar.Sign)/(r32**order-desiredVar.Sign))
ap = np.abs(np.log(np.abs(desiredVar['e32']/desiredVar['e21'])+q))/np.log(r21)
orderError = order - ap

desiredVar['Aparent Order'] = ap
desiredVar['Optimized Order'] = order
desiredVar['Order Error'] = orderError

# Extrapolated values
ext21 = ((r21**ap)*desiredVar.Variable_finer-desiredVar.Variable_medium)/((r21**ap)-1)
ext32 = ((r32**ap)*desiredVar.Variable_medium-desiredVar.Variable_coarser)/((r32**ap)-1)

desiredVar['Extrapolated Value (Finer, Medium)'] = ext21
desiredVar['Extrapolated Value (Medium, Coarser)'] = ext32

# Calculate and report the error estimatives
apxRelErr = np.abs((desiredVar.Variable_finer-desiredVar.Variable_medium)/desiredVar.Variable_finer)
extRelErr = np.abs((ext21-desiredVar.Variable_finer)/ext21)
gci = (1.25*apxRelErr)/((r21**ap)-1)

desiredVar['Aproximated Relative Error'] = apxRelErr
desiredVar['Extrapolated Relative Error'] = extRelErr
desiredVar['Grid Convergence Index'] = gci

# Export generated table
desiredVar.to_excel("treatment/results/gci.xlsx")  
\end{lstlisting}
\section{plot.py}
And finally the ploting and the spreadsheet containing the results are output by bin/plot.py file: 
\begin{lstlisting}[language=python]
#!/usr/bin/env python3
# -*- coding: utf-8 -*-

import matplotlib.pyplot as plt

fig, ax = plt.subplots(figsize=(9,6), dpi=300)
ax.plot(desiredVar.index, desiredVar.Variable_coarser,
            label= 'Coarser', aa=True)
ax.plot(desiredVar.index, desiredVar.Variable_medium,
            label= 'Medium', aa=True)
ax.plot(desiredVar.index, desiredVar.Variable_finer,
            label= 'Finer', aa=True)

ax.legend(loc='best',fontsize='x-large')

plt.grid()
plt.autoscale(enable=True, tight=True)
plt.savefig('treatment/results/allMeshes.png')

fig, ax = plt.subplots(figsize=(9,6), dpi=300)
ax.errorbar(desiredVar.index, desiredVar.Variable_medium,
            gci*desiredVar.Variable_medium,
            errorevery = 5, elinewidth = 1,
            uplims = True, lolims = True, 
            lw=1.5, aa = True)

plt.grid()
plt.autoscale(enable=True, tight=True)
plt.savefig('treatment/results/mediumWithErrorbars.png')

fig, ax = plt.subplots(figsize=(9,6), dpi=300)
ax.errorbar(desiredVar.index, desiredVar.Variable_finer,
            gci*desiredVar.Variable_finer,
            errorevery = 5, elinewidth = 1,
            uplims = True, lolims = True, 
            lw=1.5, aa = True)

plt.grid()
plt.autoscale(enable=True, tight=True)
plt.savefig('treatment/results/finerWithErrorbars.png')
\end{lstlisting}
\end{document}