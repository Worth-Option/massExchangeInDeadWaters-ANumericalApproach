\chapter*{Abstract}
\pagenumbering{roman}
The hydrodynamics of dead waters (DZ) were investigated for two different types of structures: lateral cavities and groyne fields. A literature review of the main methods of investigation of this kind of flow was conducted in which knowledge gaps were identified. The structure of this dissertation starts with a numerical model of groyne fields that identified different phases in the mass exchange between the DZ and the main channel. Following, a numerical model was developed to describe the hydrodynamics of a lateral cavity using a hybrid method to account for the turbulence fields (Detached Eddy Simulation) under a commercial package. This model was further developed in a Large Eddy Simulation (LES) under an open-source package to make the data accessible. Lastly, the main topic of this dissertation was described in which the investigation of a vegetated lateral cavity was investigated. In this paper, we found the presence of a secondary circulation that was not expected for this geometry in a non-vegetated scenario. This discovery led us to an analysis of the flow and its variation in different vegetation densities to discuss and find thresholds for the vegetation density.