\chapter*{Resumo}
\pagenumbering{roman}
A hidrodinâmica de zonas mortas foi investigada em dois diferentes tipos de estruturas: cavidades laterais e campos de espigão. Uma revisão de literatura dos principais métodos de investigação deste tipo de escoamento foi conduzida na qual identificamos lácunas a serem preenchidas. A estrutura desta dissertação começa com um modelo numérico de campos de espigão que identificou diferentes fazes na qual a troca de massa entre o canal inalterado e a zona morta ocorre. Em seguida, um modelo numérico foi desenvolvido para descrever a hidrodinâmica de uma cavidade lateral usando um método híbrido para calcular os campos turbulentos (\textit{Detached Eddy Simulation}) sob um pacote comercial. Este modelo foi melhorado no seguinte capitulo em uma simulação que considera os campos instantâneos do escoamento (modelo de turbulencia \textit{Large Eddy Simulation}) na qual um pacote de código aberto foi utilizado para uma ampliação do acesso do modelo. Finalmente, o principal tópico da dissertação foi descrito e consiste na investigação de uma cavidade lateral vegetada. Neste artigo, descobrimos a presença de uma circulação secundária que não era esperada para essa geometria, caso não houvesse vegetação. Aleḿ disso, o artigo trata da análise do escoamento e sua variação em diferentes níveis de densidade de vegetação a qual nos levou a encontrar um valor limite que divide o escoamento.
